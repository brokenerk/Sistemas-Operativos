\documentclass[12pt]{article}
\usepackage[utf8]{inputenc}
\usepackage[spanish]{babel}
\usepackage{bbding}
\decimalpoint
\usepackage[spanish]{babel}
\usepackage{amsmath}
\usepackage{amsthm}
\usepackage{amssymb}
\usepackage{graphicx}
\usepackage[margin=0.9in]{geometry}
\usepackage{fancyhdr}
\usepackage[inline]{enumitem}
\usepackage{float}
\usepackage{cancel}
\usepackage{minted}
\usepackage{bigints}
\usepackage{color}
\usepackage{xcolor}
\usepackage{subfig}
\usepackage{listingsutf8}
\usepackage{algorithm}
\usepackage{tocloft}
\usepackage[none]{hyphenat}
\usepackage{graphicx}
\usepackage{grffile}
\usepackage{tabularx}
\usepackage[nottoc,notlot,notlof]{tocbibind}
\usepackage{times}
\usepackage{color}
\definecolor{gray97}{gray}{.97}
\definecolor{gray75}{gray}{.75}
\definecolor{gray45}{gray}{.45}
\renewcommand{\cftsecleader}{\cftdotfill{\cftdotsep}}
\pagestyle{fancy}
\setlength{\headheight}{15pt} 
\lhead{PRACTICA 3: Introducción a los sistemas operativos Linux y Windows (3)}
\rhead{\thepage}
\lfoot{ESCOM-IPN}
\renewcommand{\footrulewidth}{0.5pt}
\setlength{\parskip}{0.5em}
\newcommand{\ve}[1]{\overrightarrow{#1}}
\newcommand{\abs}[1]{\left\lvert #1 \right\lvert}
\date{26 de febrero de 2017}
\title{Instalación de Netbeans}
\author{Reporte 1}

\definecolor{pblue}{rgb}{0.13,0.13,1}
\definecolor{pgreen}{rgb}{0,0.5,0}
\definecolor{pred}{rgb}{0.9,0,0}
\definecolor{pgrey}{rgb}{0.46,0.45,0.48}
\lstset{tabsize=1}


\lstset{ frame=Ltb,
framerule=0pt,
aboveskip=0.5cm,
framextopmargin=3pt,
framexbottommargin=3pt,
framexleftmargin=0.4cm,
framesep=0pt,
rulesep=.4pt,
backgroundcolor=\color{gray97},
rulesepcolor=\color{black},
%
stringstyle=\ttfamily,
showstringspaces = false,
basicstyle=\small\ttfamily,
commentstyle=\color{gray45},
keywordstyle=\bfseries,
%
numbers=left,
numbersep=15pt,
numberstyle=\tiny,
numberfirstline = false,
breaklines=true,
}

% minimizar fragmentado de listados
\usepackage{listings}
\lstnewenvironment{listing}[1][]
{\lstset{#1}\pagebreak[0]}{\pagebreak[0]}

\lstdefinestyle{consola}
{basicstyle=\scriptsize\bf\ttfamily,
backgroundcolor=\color{gray75},
}

\lstdefinestyle{C}
{language=C,
}
 \lstset{style=CompilandoStyle}                                  %Use this style

    \usepackage{minted} % Paquete que permite citar codigo
    \usemintedstyle{borland} % Aqui se define el colorscheme para minted
    \setminted{
        fontsize = \scriptsize, % Ajusta el codigo a la hoja
        baselinestretch = 1,
        linenos, % set numbers
        breaklines=true, % Hace un salto de linea automatico en caso de que se llege al final de la line
        tabsize=3 
    }
%%%%%%%%%%%%%%%%%%%%%

\lstdefinestyle{customc}{
  belowcaptionskip=1\baselineskip,
  breaklines=true,
  frame=L,
  xleftmargin=\parindent,
  language=C,
  showstringspaces=false,
  basicstyle=\footnotesize\ttfamily,
  keywordstyle=\bfseries\color{green!40!black},
  commentstyle=\itshape\color{purple!40!black},
  identifierstyle=\color{blue},
  stringstyle=\color{orange},
}

\lstdefinestyle{customasm}{
  belowcaptionskip=1\baselineskip,
  frame=L,
  xleftmargin=\parindent,
  language=[x86masm]Assembler,
  basicstyle=\footnotesize\ttfamily,
  commentstyle=\itshape\color{purple!40!black},
}

\lstset{escapechar=@,style=customc}

    % =====  CODE EDITOR =========
    \lstdefinestyle{CompilandoStyle} {                              %This is Code Style
        backgroundcolor=\color{BlueGrey800MD},                      %Background Color  
        basicstyle=\tiny\color{white},                              %Font color
        commentstyle=\color{BlueGrey100MD},                         %Comment color
        stringstyle=\color{TealMD},                                 %String color
        keywordstyle=\color{Green100MD},                            %keywords color
        numberstyle=\tiny\color{TealMD},                            %Size of a number
        frame=shadowbox,                                            %Adds a frame around the code
        breakatwhitespace=true,                                     %Style                       
        breaklines=true,                                            %Style                   
        keepspaces=true,                                            %Style                   
        numbers=left,                                               %Style                   
        numbersep=10pt,                                             %Style 
        xleftmargin=\parindent,                                     %Style 
        tabsize=4                                                   %Style 
    }
 
    \lstset{style=CompilandoStyle}                                  %Use this style

    \usepackage{minted} % Paquete que permite citar codigo
    \usemintedstyle{borland} % Aqui se define el colorscheme para minted
    \setminted{
        fontsize = \scriptsize, % Ajusta el codigo a la hoja
        baselinestretch = 1,
        linenos, % set numbers
        breaklines=true, % Hace un salto de linea automatico en caso de que se llege al final de la line
        tabsize=3 
    }


%Permite crear columnas en el documento
\usepackage{multicol} 
\usepackage{color}
\usepackage{comment}
\newcommand{\tabitem}{~~\llap{\textbullet}~~}
\newcommand{\subtabitem}{~~~~\llap{\textbullet}~~}
\usepackage{minted}

\bibliographystyle{IEEEtran}
\begin{document}

		\begin{titlepage}
			\begin{center}
				
				% Upper part of the page. The '~' is needed because \\
				% only works if a paragraph has started.
				
				\noindent
				\begin{minipage}{0.5\textwidth}
					\begin{flushleft} \large
						\includegraphics[width=0.3\textwidth]{../ipn.png}
					\end{flushleft}
				\end{minipage}%
				\begin{minipage}{0.55\textwidth}
					\begin{flushright} \large
						\includegraphics[width=0.7\textwidth]{../escom.png}
					\end{flushright}
				\end{minipage}
				
				\textsc{\LARGE Instituto Politécnico Nacional}\\[0.5cm]
				
				\textsc{\Large Escuela Superior de Cómputo}\\[1cm]
				
				% Title
				
				{ \huge Práctica 4 - Administrador de procesos en Linux y Windows (3) \\[1cm] }
				
				{ \Large Unidad de aprendizaje: Sistemas Operativos} \\[1cm]
				
				{ \Large Grupo: 2CM8 } \\[1cm]
				
				\noindent
				\begin{minipage}{0.5\textwidth}
					\begin{flushleft} \large
						\emph{Alumnos(a):}\\
						
						\begin{tabular}{ll}
						 Méndez Mejía Sergio Ernesto \\
					     Nicolás Sayago Abigail\\
					     Ramos Diaz Enrique \\
					     Mariana \\
					     
					\end{tabular}
					\end{flushleft}
				\end{minipage}%
				\begin{minipage}{0.5\textwidth}
					\begin{flushright} \large
						\emph{Profesor(a):} \\
						Cortes Galicia Jorge  \\
					\end{flushright}
				\end{minipage}
				
				\vfill
				
				% Bottom of the page
				{\large 22 de Octubre 2018}
			\end{center}
		\end{titlepage}
	
	\tableofcontents
	\newpage

% //////////////////////////////////////////////////////////////////////////////////////////////////////////////
%                                                   COMPETENCIAS
% /////////////////////////////////////////////////////////////////////////////////////////////////////////////

	\section{Competencias}
	El alumno aprende a familiarizarse con el administrados de procesos del sistema operativo Linux y Windows a través de la creación de nuevos procesos por copia exacta de código y/o por sustitución de código para el desarrollo de aplicaciones concurrentes sencillas.
    \begin{itemize}
        \item[\Checkmark] Revisión de la creación de procesos en Linux y Windows.

        \item[\Checkmark] Revisión de las llamadas al sistema para la creación de procesos en Linux y Windows.

        \item[\Checkmark] Desarrollo de aplicaciones concurrentes mediante la creación de procesos en ambos sistemas operativos.
    \end{itemize}
	
% //////////////////////////////////////////////////////////////////////////////////////////////////////////////
%                                                   DESARROLLO
% /////////////////////////////////////////////////////////////////////////////////////////////////////////////
	
    \section{Desarrollo}
    \subsection{Puntos a observar y reportar}
            
            % ///////////////////////////////////////////////////////////////////////////////////////////////
            %                              COMANDOS DE LINUX
            % ///////////////////////////////////////////////////////////////////////////////////////////////
        
            \subsubsection{Sección Linux:}
                \begin{itemize}
                    %  PUNTO 1 - Introduzca los siguientes comandos a través de la consola del sistema operativo Linux: 
        	        \item[\Checkmark] \textbf{Prueba de comandos}
        	            \begin{itemize}
        	                \item \textbf{ps}
        	                    \begin{figure}[h!]
                                    \centering
                                    \includegraphics[width=0.8\textwidth]{Practica4/Images/Linux/NOMBRE.png}
                                        \caption{Prueba del comando ps}
                                \end{figure}
                                Información que proporciona el comando ejecutado:
                        
        	                \item \textbf{ps -fea}
        	                    \begin{figure}[h!]
                                    \centering
                                    \includegraphics[width=0.8\textwidth]{Practica4/Images/Linux/NOMBRE.png}
                                        \caption{Prueba del comando ps}
                                \end{figure}
                                Información que proporciona el comando ejecutado:
        	            \end{itemize}
        	       % PUNTO 2 - A través de la ayuda en linea que proporciona Linux, investigue para que se utiliza el comando ps y mencione las opciones que se pueden utilizar con dicho comando. Además investigue el uso de las llamadas al sistema fork(), execv(), getpid(), getppid() y wait() en la ayuda en líena, mencione que otras funciones similares a execv() existen, reporte sus observaciones
        	       \item[\Checkmark] \textbf{Investigación de comandos}
                    \begin{itemize}
                        \item \textbf{ps}

                        \item \textbf{fork()}

                        \item \textbf{execv()}

                        \item \textbf{getpid()}

                        \item \textbf{getppid()}

                        \item \textbf{wait()}

                    \end{itemize}
            	\end{itemize}
        
            % ///////////////////////////////////////////////////////////////////////////////////////////////
            %                              SECCIÓN WINDOWS
            % ///////////////////////////////////////////////////////////////////////////////////////////////
    
            \subsubsection{Sección Windows:}
                \begin{itemize}
                    \item[\Checkmark] \textbf{Punto 6:}
                     Diferencias y similitudes de creación de código en Linux y Windows.

                     % EXPLICAR AQUÍ

        	        \item[\Checkmark] \textbf{Punto 7:} Función \textbf{GetCurrentProcessId()}
        	       % AQUI PONER INVESTIGACIÓN
            	\end{itemize}
        
        % ////////////////////////////////////////////////////////////////////////////////////////////////////
        % 						CODIGOS FUENTE DE LOS PROGRAMAS DESARROLLADOS
        % ////////////////////////////////////////////////////////////////////////////////////////////////////

    	\subsection{Códigos fuente de los programas desarrollados}
    	% AQUI SOLO VA EL CODIGO FUENTE, MÁS ABAJO EXPLICAN LO QUE QUIERAN EN LA PARTE DE EJECUCIÓN :)
    	\subsubsection{Sección Linux}

        	\begin{itemize}
        	    \item[\Checkmark] \textbf{Punto 3 :}
                    \begin{itemize}
                        \item Primer código 
                   	       \inputminted{octave}{Code/Linux/3_1.c}
                        \item Segundo código
                           \inputminted{octave}{Code/Linux/3_2.c}
                    \end{itemize}

                \item[\Checkmark] \textbf{Punto 4 :}

                    Crea un árbol de procesos mostrado en el pizarrón. Para cada uno de los procesos creados(por copia exacta de código) se imprimirá en pantalla el pid de su padre si se trata de un hijo terminal o los pid's de sus hijos creados si se trata de un proceso padre. 
        	        \inputminted{octave}{Code/Linux/4.c}
                
                \item[\Checkmark] \textbf{Punto 5 :}
        	        
                    Aplicación que crea seis procesos (por cópia exacta de código). El primer proceso se encarga de realizar la suma de dos matrices $10$ $x$ $10$ elementos tipo entero, el segundo proceso realizará la resta sobre esas mismas matrices, el tercer proceso realizará la multiplicación de las matrices, el cuarto proceso obtendrá las tranaspuestas de cada matriz y elq uinto proceso obtendrá las matrices inversas. Cada uno de estos procesos escribirá un archivo con los resultados de la operación que realizo. El sexto proceso leerán los archivos de resultados y los mostrará en pantalla cada uno de ellos.

                    \begin{itemize}
                        \item \textbf{Aplicación secuencial}
                            \inputminted{octave}{Code/Linux/5_1.c}
                        \item \textbf{Aplicación con procesos}
                            \inputminted{octave}{Code/Linux/5_2.c}  
                    \end{itemize}
                
                \item[\Checkmark] \textbf{Punto 6 :}
        	        
        	        Crea un nuevo proceso con sustitución de un nuevo código, así como el programa que será el nuevo código a ejecutar.
        	        \begin{itemize}
                        \item \textbf{Código de sustitución}
                            \inputminted{octave}{Code/Linux/6_1.c}
                        \item \textbf{Código a ejecutar}
                            \inputminted{octave}{Code/Linux/6_2.c}  
                    \end{itemize}
                
                \item[\Checkmark] \textbf{Punto 7 :}

                    Aplicación que creea un proceso hijo a partir de un proceso padre, el hijo creado a su vez creará tres procesos hijos más. Cada uno de los tres procesos generados ejecutará tres programas diferentes mediante sustitución de código, el primer programa evaluará una expresión aritmética, el segundo programa cambiará los permisos de un archivo dado, y el tercer programa obtendrá las matrices inversas.

                    \begin{itemize}
                        \item \textbf{Código de procesos}
                            \inputminted{octave}{Code/Linux/7.c}
                        
                        \item \textbf{Código expresión aritmetica}
                            \inputminted{octave}{Code/Linux/expresion.c}

                        \item \textbf{Código de permisos}
                            \inputminted{octave}{Code/Linux/permisos.c}

                        \item \textbf{Código matriz inversa}
                            \inputminted{octave}{Code/Linux/inversa.c}  
                    \end{itemize}

                \item[\Checkmark] \textbf{Punto 8 :}
                    
                    Aplicación que crea seis procesos (por sustitución de código). El primer proceso se encarga de realizar la suma de dos matrices $10$ $x$ $10$ elementos tipo entero, el segundo proceso realizará la resta sobre esas mismas matrices, el tercer proceso realizará la multiplicación de las matrices, el cuarto proceso obtendrá las tranaspuestas de cada matriz y elq uinto proceso obtendrá las matrices inversas. Cada uno de estos procesos escribirá un archivo con los resultados de la operación que realizo. El sexto proceso leerán los archivos de resultados y los mostrará en pantalla cada uno de ellos.

                    \begin{itemize}
                        \item \textbf{Aplicación con procesos}
                            \inputminted{octave}{Code/Linux/8.c}
                        \item \textbf{Código suma de matrices}
                            \inputminted{octave}{Code/Linux/suma.c}

                        \item \textbf{Código resta de matrices}
                            \inputminted{octave}{Code/Linux/suma.c}

                        \item \textbf{Código multiplicación de matrices}
                            \inputminted{octave}{Code/Linux/suma.c}  

                        \item \textbf{Código transpuesta de una matriz}
                            \inputminted{octave}{Code/Linux/suma.c}

                        \item \textbf{Código de inversa}
                            \inputminted{octave}{Code/Linux/suma.c}

                        % Poner los que sean necesarios
                    \end{itemize}
        	\end{itemize}
    	
    	\subsubsection{Sección Windows}
    	
    	\begin{itemize}
    	    \item[\Checkmark] \textbf{Punto 3 :}  
                Programa de creación de un nuevo proceso.   
        	    \inputminted{octave}{Code/Windows/3.c}

            \item[\Checkmark] \textbf{Punto 4 :}
                Programa que contrendra al proceso hijo.     
                \inputminted{octave}{Code/Windows/4.c}
                
            \item[\Checkmark] \textbf{Punto 5 :}
                Programa que contrendra al proceso hijo, con un nuevo argumento.
                \inputminted{octave}{Code/Windows/5.c}

            \item[\Checkmark] \textbf{Punto 7 :}
                Aplicación que crea un proceso hijo a partir de un proceso padre, el hijo creado a su vez creará 5 procesos hijos más. A su vez cada uno de los cinco procesos creará 3 procesos más. Cada uno de los procesos creados imprimirá en pantalla su identificador.
                \inputminted{octave}{Code/Windows/6.c}

            \item[\Checkmark] \textbf{Punto 8 :}
                 Aplicación que crea seis procesos (por cópia exacta de código). El primer proceso se encarga de realizar la suma de dos matrices $10$ $x$ $10$ elementos tipo entero, el segundo proceso realizará la resta sobre esas mismas matrices, el tercer proceso realizará la multiplicación de las matrices, el cuarto proceso obtendrá las tranaspuestas de cada matriz y elq uinto proceso obtendrá las matrices inversas. Cada uno de estos procesos escribirá un archivo con los resultados de la operación que realizo. El sexto proceso leerán los archivos de resultados y los mostrará en pantalla cada uno de ellos.

                \begin{itemize}
                    \item \textbf{Aplicación secuencial}
                        \inputminted{octave}{Code/Windows/8_1.c}
                    \item \textbf{Aplicación con procesos}
                        \inputminted{octave}{Code/Windows/8_2.c}  
                \end{itemize}
            
    	\end{itemize}
    	
        % ////////////////////////////////////////////////////////////////////////////////////////////////////
        %                               PANTALLAS DE EJECUCIÓN DE LOS PROGRAMAS DESARROLLADOS
        % ////////////////////////////////////////////////////////////////////////////////////////////////////
        
        \newpage
    	\subsection{Pantallas de ejecución de los programas desarrollados}

    		% ///////////////////////////////////////////////////////////////////////////////////////////////
	        % 											SECCION LINUX
	        % ///////////////////////////////////////////////////////////////////////////////////////////////
                        
            % AQUI SI VAN LAS EXPLICACIONES QUE SEAN NECESARIAS	        
    		\subsubsection{Sección Linux:}
    		\begin{itemize}
                \item[\Checkmark] \textbf{Punto 3 :}
                    \begin{itemize}
                        \item Primer código 
                           \begin{figure}[h!]
                                \centering
                                \includegraphics[width=0.8\textwidth]{Practica4/Images/Linux/NOMBRE.png}
                                \caption{DESCRIPCIÓN}
                            \end{figure}

                            % AQUI PUEDES PONER COMENTARIOS QUE SEAN NECESARIOS
                            % TAMBIEN PUEDES AGREGAR LAS IMAGENEES QUE QUIERAS PARA MOSTRAR QUE EXPERIMENTASTE COMO DICE LA PRACTICA

                        \item Segundo código
                            \begin{figure}[h!]
                                \centering
                                \includegraphics[width=0.8\textwidth]{Practica4/Images/Linux/NOMBRE.png}
                                \caption{DESCRIPCIÓN}
                            \end{figure}

                            % AQUI PUEDES PONER COMENTARIOS QUE SEAN NECESARIOS
                    \end{itemize}

                \item[\Checkmark] \textbf{Punto 4 :}

                    Crea un árbol de procesos mostrado en el pizarrón. Para cada uno de los procesos creados(por copia exacta de código) se imprimirá en pantalla el pid de su padre si se trata de un hijo terminal o los pid's de sus hijos creados si se trata de un proceso padre. 

                    \begin{figure}[h!]
                        \centering
                        \includegraphics[width=0.8\textwidth]{Practica4/Images/Linux/NOMBRE.png}
                        \caption{DESCRIPCIÓN}
                    \end{figure}

                    % AQUI PUEDES PONER COMENTARIOS QUE SEAN NECESARIOS
                
                \item[\Checkmark] \textbf{Punto 5 :}
                    
                    Aplicación que crea seis procesos (por cópia exacta de código). El primer proceso se encarga de realizar la suma de dos matrices $10$ $x$ $10$ elementos tipo entero, el segundo proceso realizará la resta sobre esas mismas matrices, el tercer proceso realizará la multiplicación de las matrices, el cuarto proceso obtendrá las tranaspuestas de cada matriz y elq uinto proceso obtendrá las matrices inversas. Cada uno de estos procesos escribirá un archivo con los resultados de la operación que realizo. El sexto proceso leerán los archivos de resultados y los mostrará en pantalla cada uno de ellos.

                    \begin{itemize}
                        \item \textbf{Aplicación secuencial}
                            \begin{figure}[h!]
                                \centering
                                \includegraphics[width=0.8\textwidth]{Practica4/Images/Linux/NOMBRE.png}
                                \caption{DESCRIPCIÓN}
                            \end{figure}

                            % AQUI PUEDES PONER COMENTARIOS QUE SEAN NECESARIOS
                        
                        \item \textbf{Aplicación con procesos}
                            \begin{figure}[h!]
                                \centering
                                \includegraphics[width=0.8\textwidth]{Practica4/Images/Linux/NOMBRE.png}
                                \caption{DESCRIPCIÓN}
                            \end{figure}

                            % AQUI PUEDES PONER COMENTARIOS QUE SEAN NECESARIOS
                          
                    \end{itemize}
                
                \item[\Checkmark] \textbf{Punto 6 :}
                    
                    Crea un nuevo proceso con sustitución de un nuevo código, así como el programa que será el nuevo código a ejecutar.
                    \begin{itemize}
                        \item \textbf{Código de sustitución}
                            \begin{figure}[h!]
                                \centering
                                \includegraphics[width=0.8\textwidth]{Practica4/Images/Linux/NOMBRE.png}
                                \caption{DESCRIPCIÓN}
                            \end{figure}

                            % AQUI PUEDES PONER COMENTARIOS QUE SEAN NECESARIOS
                        
                        \item \textbf{Código a ejecutar}
                            \begin{figure}[h!]
                                \centering
                                \includegraphics[width=0.8\textwidth]{Practica4/Images/Linux/NOMBRE.png}
                                \caption{DESCRIPCIÓN}
                            \end{figure}

                            % AQUI PUEDES PONER COMENTARIOS QUE SEAN NECESARIOS
                          
                    \end{itemize}
                
                \item[\Checkmark] \textbf{Punto 7 :}

                    Aplicación que creea un proceso hijo a partir de un proceso padre, el hijo creado a su vez creará tres procesos hijos más. Cada uno de los tres procesos generados ejecutará tres programas diferentes mediante sustitución de código, el primer programa evaluará una expresión aritmética, el segundo programa cambiará los permisos de un archivo dado, y el tercer programa obtendrá las matrices inversas.

                    \begin{figure}[h!]
                        \centering
                        \includegraphics[width=0.8\textwidth]{Practica4/Images/Linux/NOMBRE.png}
                        \caption{DESCRIPCIÓN}
                    \end{figure}

                    % AQUI PUEDES PONER COMENTARIOS QUE SEAN NECESARIOS
                                

                \item[\Checkmark] \textbf{Punto 8 :}
                    
                    Aplicación que crea seis procesos (por sustitución de código). El primer proceso se encarga de realizar la suma de dos matrices $10$ $x$ $10$ elementos tipo entero, el segundo proceso realizará la resta sobre esas mismas matrices, el tercer proceso realizará la multiplicación de las matrices, el cuarto proceso obtendrá las tranaspuestas de cada matriz y elq uinto proceso obtendrá las matrices inversas. Cada uno de estos procesos escribirá un archivo con los resultados de la operación que realizo. El sexto proceso leerán los archivos de resultados y los mostrará en pantalla cada uno de ellos.

                    \begin{figure}[h!]
                        \centering
                        \includegraphics[width=0.8\textwidth]{Practica4/Images/Linux/NOMBRE.png}
                        \caption{DESCRIPCIÓN}
                    \end{figure}

                    % AQUI PUEDES PONER COMENTARIOS QUE SEAN NECESARIOS
            \end{itemize}
        
			 
			% ///////////////////////////////////////////////////////////////////////////////////////////////
	        % 											SECCION WINDOWS
	        % ///////////////////////////////////////////////////////////////////////////////////////////////			    
	\newpage    
	        
        \begin{itemize}
            \item[\Checkmark] \textbf{Punto 3 :}  
                Programa de creación de un nuevo proceso.   
                \begin{figure}[h!]
                    \centering
                    \includegraphics[width=0.8\textwidth]{Practica4/Images/Windows/NOMBRE.png}
                    \caption{DESCRIPCIÓN}
                \end{figure}

                % AQUI PUEDES PONER COMENTARIOS QUE SEAN NECESARIOS

            \item[\Checkmark] \textbf{Punto 4 :}
                Programa que contrendra al proceso hijo.     
                \begin{figure}[h!]
                    \centering
                    \includegraphics[width=0.8\textwidth]{Practica4/Images/Windows/NOMBRE.png}
                    \caption{DESCRIPCIÓN}
                \end{figure}

                % AQUI PUEDES PONER COMENTARIOS QUE SEAN NECESARIOS
                
            \item[\Checkmark] \textbf{Punto 5 :}
                Programa que contrendra al proceso hijo, con un nuevo argumento.
                
                \begin{figure}[h!]
                    \centering
                    \includegraphics[width=0.8\textwidth]{Practica4/Images/Windows/NOMBRE.png}
                    \caption{DESCRIPCIÓN}
                \end{figure}

                % AQUI PUEDES PONER COMENTARIOS QUE SEAN NECESARIOS
            \item[\Checkmark] \textbf{Punto 7 :}
                Aplicación que crea un proceso hijo a partir de un proceso padre, el hijo creado a su vez creará 5 procesos hijos más. A su vez cada uno de los cinco procesos creará 3 procesos más. Cada uno de los procesos creados imprimirá en pantalla su identificador.

                \begin{figure}[h!]
                    \centering
                    \includegraphics[width=0.8\textwidth]{Practica4/Images/Windows/NOMBRE.png}
                    \caption{DESCRIPCIÓN}
                \end{figure}

                % AQUI PUEDES PONER COMENTARIOS QUE SEAN NECESARIOS

            \item[\Checkmark] \textbf{Punto 8 :}
                 Aplicación que crea seis procesos (por cópia exacta de código). El primer proceso se encarga de realizar la suma de dos matrices $10$ $x$ $10$ elementos tipo entero, el segundo proceso realizará la resta sobre esas mismas matrices, el tercer proceso realizará la multiplicación de las matrices, el cuarto proceso obtendrá las tranaspuestas de cada matriz y elq uinto proceso obtendrá las matrices inversas. Cada uno de estos procesos escribirá un archivo con los resultados de la operación que realizo. El sexto proceso leerán los archivos de resultados y los mostrará en pantalla cada uno de ellos.

                \begin{itemize}
                    \item \textbf{Aplicación secuencial}
                        \begin{figure}[h!]
                            \centering
                            \includegraphics[width=0.8\textwidth]{Practica4/Images/Windows/NOMBRE.png}
                            \caption{DESCRIPCIÓN}
                        \end{figure}

                        % AQUI PUEDES PONER COMENTARIOS QUE SEAN NECESARIOS

                    \item \textbf{Aplicación con procesos}
                        \begin{figure}[h!]
                            \centering
                            \includegraphics[width=0.8\textwidth]{Practica4/Images/Windows/NOMBRE.png}
                            \caption{DESCRIPCIÓN}
                        \end{figure}

                        % AQUI PUEDES PONER COMENTARIOS QUE SEAN NECESARIOS  
                \end{itemize}
            
        \end{itemize}
        
         
% //////////////////////////////////////////////////////////////////////////////////////////////////////////////
%                                                   OBSERVACIONES
% /////////////////////////////////////////////////////////////////////////////////////////////////////////////
    \newpage
	\section{Observaciones}
        \begin{itemize}
            \item[\Checkmark]
        
            \item[\Checkmark] 
        \end{itemize}

% //////////////////////////////////////////////////////////////////////////////////////////////////////////////
%                                                  ANALISIS CRITICO
% /////////////////////////////////////////////////////////////////////////////////////////////////////////////
	
	\section{Análisis Crítico}
	
% //////////////////////////////////////////////////////////////////////////////////////////////////////////////
%                                                   	CONCLUSIONES
% /////////////////////////////////////////////////////////////////////////////////////////////////////////////
	
	\section{Conclusiones}
    
\end{document}
	                                                                                                                                                                                                                                                                                                                                       